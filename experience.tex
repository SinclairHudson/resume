%----------------------------------------------------------------------------------------
%	EXPERIENCE SECTION
%----------------------------------------------------------------------------------------

\roottitle{Relevant Work Experience} % Top level section

\headedsection % Employer name which can include a hyperlink and location/URL on the right side of the page
{\href{http://www.getcruise.com}{Cruise}}
{\textsc{San Francisco, CA}} {
\headedsubsection % Job title entry for the current employer
{Model Deployment Platform Engineer}
{Sept '22 -- Dec '22}
{\bodytext{
\begin{itemize}
%\item Developed various Python tools to deploy deep learning models on the vehicle, using \href{https://onnx.ai/}{ONNX},
   % \href{https://developer.nvidia.com/tensorrt}{TensorRT}, and custom computation graph representations.
\item Developed debugging tools to assist in model deployment, reducing deployment time by days in some cases.
\item Built a tool to automatically assess numerical divergence between the original PyTorch models and optimized 
    \href{https://developer.nvidia.com/tensorrt}{TensorRT} models, allowing deployment engineers to quickly identify optimization errors.
\item Built a linter to map \href{https://onnx.ai/}{ONNX} nodes to lines in PyTorch source code, saving deployment engineers hours of manual debugging every deployment.
\item Created a tool to manually add outputs to exported ONNX graphs, allowing engineers to inspect intermediate activations while debugging models.
\end{itemize}

}}
}

\headedsection % Employer name which can include a hyperlink and location/URL on the right side of the page
{\href{http://www.nvidia.com}{NVIDIA}}
{\textsc{Santa Clara, CA (remote)}} {
\headedsubsection % Job title entry for the current employer
{Deep Learning Researcher for Autonomous Vehicles}
{Jan '22 -- Apr '22}
{\bodytext{
\begin{itemize}
\item Designed and iterated on multiple experiments for a LiDAR object detection neural network, improving cyclist and pedestrian detections by 43\% and 15\%, respectively. 
\item Implemented sparse tensor object detectors using \href{https://github.com/NVIDIA/MinkowskiEngine}{Minkowski Engine}, outperforming the baseline model while using 70\% less memory.
\item Presented a 30-minute research overview to the organization, summarizing research findings of the whole term.
\item Integrated confidence predictions into a LiDAR object detection auto-labeling pipeline, allowing human annotators to focus efforts on anomalous and challenging data instances.
%\item Consistently deployed and monitored experiments on internal GPU clusters, maintaining high GPU utilization throughout the internship.
\end{itemize}

}}
}

\headedsection % Employer name which can include a hyperlink and location/URL on the right side of the page
{\href{https://darwinai.com/}{DarwinAI}}
{\textsc{Waterloo, ON (remote)}} {

\headedsubsection % Job title entry for the current employer
{Machine Learning Developer}
{May '21 -- Aug '21}
{\bodytext{
\begin{itemize}
\item Built and tested defect detection deep learning solutions for clients in the manufacturing industry.
\item Trained XGBoost and SVR systems to model the relationship between environmental conditions and yield for an agriculture client,
    giving the client insight into the impact of environmental conditions on their yield.
\item Implemented the core functionality of Dataset Distillation using the autograd package, to pursue research in low-data machine learning contexts.
%\item Designed and created an anomaly detection research repository in PyTorch, for internal research detecting anomalies in images.
\item Implemented VAE, VQ-VAE, and VQ-VAE-2 from scratch in PyTorch, demonstrating the 
    effectiveness of autoencoders as anomaly detectors on client datasets.
\end{itemize}

}}
}

\headedsection % Employer name which can include a hyperlink and location/URL on the right side of the page
 {\href{https://www.untether.ai/}{Untether AI}}
 {\textsc{Toronto, ON (remote)}} {

\headedsubsection % Job title entry for the current employer
{Software Developer}
{Sept '20 -- Dec '20}
{\bodytext{
\begin{itemize}
\item Contributed to a customer-facing Python library to quantize and optimize TensorFlow computation graphs.
\item Designed and implemented Non-Max Suppression for quantized values using only integer operations, allowing
    object detection pipelines to run 10x faster on-chip.
\item Experimented with different quantization schemes to improve the mAP of an SSD-ResNet-34 by 5\%.
 \item Implemented a lookup table class to represent arbitrary non-linear functions in a quantized space, 
     allowing the python library to support any activation function.
 \end{itemize}

 }}
 }

\headedsection % Employer name which can include a hyperlink and location/URL on the right side of the page
{\href{http://dev3.noahlab.com.hk/about.html}{Huawei}}
{\textsc{Markham, ON}} {

\headedsubsection % Job title entry for the current employer
{LiDAR Perception Researcher}
{Jan '20 -- Apr '20}
{\bodytext{
\begin{itemize}
    \item Implemented key modules from 12 different research papers in PyTorch, summarizing state-of-the-art techniques and enabling further research. 
    %\item Built DBLiDARNet and focal loss from scratch in PyTorch to use in semantic segmentation experiments.
    \item Analysed the SemanticKITTI dataset to produce optimal class loss weights, boosting mIoU by 2\%.
    \item Wrote a data loader to spatially align sequential LiDAR scans for temporal pipelines, based on IMU data, enabling
        the research of multi-scan LiDAR segmentation pipelines.
    \end{itemize}
    
    }}
    }
%------------------------------------------------

 \begin{center}
\textit{Please refer to \href{https://www.linkedin.com/in/sinclairhudson/}{my Linkedin profile} for a complete list of work experiences.}
\end{center}
