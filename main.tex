%%%%%%%%%%%%%%%%%%%%%%%%%%%%%%%%%%%%%%%%%
% Cies Resume/CV
% LaTeX Template
% Version 1.1 (20/7/14)
%
% This template has been downloaded from:
% http://www.LaTeXTemplates.com
%
% Original author:
% Cies Breijs (cies@kde.nl)
% https://github.com/cies/resume with extensive modifications by:
% Vel (vel@latextemplates.com)
%
% License:
% CC BY-NC-SA 3.0 (http://creativecommons.org/licenses/by-nc-sa/3.0/)
%
%%%%%%%%%%%%%%%%%%%%%%%%%%%%%%%%%%%%%%%%%

%----------------------------------------------------------------------------------------
%	PACKAGES AND OTHER DOCUMENT CONFIGURATIONS
%----------------------------------------------------------------------------------------

\documentclass[10pt,a4paper]{article} % Font size (10-12pt) and paper size (a4paper, letterpaper, legalpaper, etc)

% Copyright (c) 2012 Cies Breijs
%
% The MIT License
%
% Permission is hereby granted, free of charge, to any person obtaining a copy
% of this software and associated documentation files (the "Software"), to deal
% in the Software without restriction, including without limitation the rights
% to use, copy, modify, merge, publish, distribute, sublicense, and/or sell
% copies of the Software, and to permit persons to whom the Software is
% furnished to do so, subject to the following conditions:
%
% The above copyright notice and this permission notice shall be included in
% all copies or substantial portions of the Software.
%
% THE SOFTWARE IS PROVIDED "AS IS", WITHOUT WARRANTY OF ANY KIND, EXPRESS OR
% IMPLIED, INCLUDING BUT NOT LIMITED TO THE WARRANTIES OF MERCHANTABILITY,
% FITNESS FOR A PARTICULAR PURPOSE AND NONINFRINGEMENT. IN NO EVENT SHALL THE
% AUTHORS OR COPYRIGHT HOLDERS BE LIABLE FOR ANY CLAIM, DAMAGES OR OTHER
% LIABILITY, WHETHER IN AN ACTION OF CONTRACT, TORT OR OTHERWISE, ARISING FROM,
% OUT OF OR IN CONNECTION WITH THE SOFTWARE OR THE USE OR OTHER DEALINGS IN THE
% SOFTWARE.

%%% LOAD AND SETUP PACKAGES

\usepackage[margin=0.5in]{geometry} % Adjusts the margins

\usepackage{multicol} % Required for multiple columns of text

\usepackage{mdwlist} % Required to fine tune lists with a inline headings and indented content

\usepackage{relsize} % Required for the \textscale command for custom small caps text

\usepackage{hyperref} % Required for customizing links
\usepackage{xcolor} % Required for specifying custom colors
\definecolor{dark-blue}{rgb}{0.15,0.15,0.4} % Defines the dark blue color used for links
\hypersetup{colorlinks,linkcolor={dark-blue},citecolor={dark-blue},urlcolor={dark-blue}} % Assigns the dark blue color to all links in the template

\usepackage{tgpagella} % Use the TeX Gyre Pagella font throughout the document
\usepackage[T1]{fontenc}
\usepackage{microtype} % Slightly tweaks character and word spacings for better typography
\usepackage{lastpage}
\usepackage{fancyhdr}

\pagestyle{fancy}
\fancyhf{}
\renewcommand{\headrulewidth}{0pt}
%\fancyfoot[R]{Page \thepage \hspace{1pt} of \pageref{LastPage}}

%----------------------------------------------------------------------------------------
%	DEFINE STRUCTURAL COMMANDS
%----------------------------------------------------------------------------------------

\newenvironment{indentsection} % Defines the indentsection environment which indents text in sections titles
{\begin{list}{}{
\setlength{\leftmargin}{10pt}
\setlength{\parsep}{0pt}
\setlength{\parskip}{0pt}
\setlength{\itemsep}{0pt}
\setlength{\topsep}{0pt}
}}{\end{list}}

\newcommand*\maintitle[2]{\noindent{\LARGE \textbf{#1}}\ \ \ \emph{#2}\vspace{0.3em}} % Main title (name) with date of birth or subtitle

\newcommand*\roottitle[1]{\subsection*{#1}\vspace{-0.6em}\nopagebreak[4]} % Top level sections in the template

\newcommand{\headedsection}[3]{\nopagebreak[4]\begin{indentsection}\item[]\textscale{1.1}{#1}\hfill#2#3\end{indentsection}\nopagebreak[4]} % Section title used for a new employer
%\newcommand{\headedsection}[3]{\nopagebreak[4]\item[]\textscale{1.1}{#1}\hfill#2#3\nopagebreak[4]} % Section title used for a new employer

% \newcommand{\headedsubsection}[3]{\nopagebreak[4]\begin{indentsection}\item[]\textbf{#1}\hfill\emph{#2}#3\end{indentsection}\nopagebreak[4]} % Section title used for a new position
\newcommand{\headedsubsection}[3]{\nopagebreak[4]\item[]\textbf{#1}\hfill\emph{#2}#3\nopagebreak[4]} % Section title used for a new position

\newcommand{\bodytext}[1]{\nopagebreak[4]\begin{indentsection}\item[]#1\end{indentsection}\pagebreak[2]} % Body text (indented)

\newcommand{\inlineheadsection}[2]{\begin{basedescript}{\setlength{\leftmargin}{\doubleparindent}}\item[\hspace{\newparindent}\textbf{#1}]#2\end{basedescript}\vspace{-1.7em}} % Section title where body text starts immediately after the title

\newcommand*\acr[1]{\textscale{.85}{#1}} % Custom acronyms command

\newcommand*\bull{\ \ \raisebox{-0.365em}[-1em][-1em]{\textscale{4}{$\cdot$}} \ } % Custom bullet point for separating content

\newlength{\newparindent} % It seems not to work when simply using \parindent...
\addtolength{\newparindent}{\parindent}

\newlength{\doubleparindent} % A double \parindent...
\addtolength{\doubleparindent}{\parindent}

\newcommand{\breakvspace}[1]{\pagebreak[2]\vspace{#1}\pagebreak[2]} % A custom vspace command with custom before and after spacing lengths
\newcommand{\nobreakvspace}[1]{\nopagebreak[4]\vspace{#1}\nopagebreak[4]} % A custom vspace command with custom before and after spacing lengths that do not break the page

\newcommand{\spacedhrule}[2]{\breakvspace{#1}\hrule\nobreakvspace{#2}} % Defines a horizontal line with some vertical space before and after it
 % Include structure.tex which contains packages and document layout definitions

\hyphenation{Some-long-word} % Specify custom hyphenation points in words with dashes where you would like hyphenation to occur, or alternatively, don't put any dashes in a word to stop hyphenation altogether

\begin{document} 

%----------------------------------------------------------------------------------------
%	NAME AND CONTACT INFORMATION
%----------------------------------------------------------------------------------------

\maintitle{Sinclair Hudson}{University of Waterloo Computer Science, class of 2023}  % Your name and date of birth or subtitle

\noindent\href{mailto:sshudson@uwaterloo.ca}{sshudson@uwaterloo.ca}\bull % Your email address
\textsmaller{+}1 (519) 694-0104 \bull % Your phone number(s) and Skype username
% \textsmaller{+}1 (628) 777-8349 \bull % Your phone number(s) and Skype username
\href{http://www.sinclairhudson.com}{sinclairhudson.com} \bull
\href{https://github.com/SinclairHudson}{GitHub: SinclairHudson}
% 123 Broadway\bull City, 12345\bull State\bull Country % Your address

\spacedhrule{0.4em}{-0.4em} % Horizontal rule - the first bracket is whitespace before and the second is after
\roottitle{Education} % Top level section

\headedsection % Employer name which can include a hyperlink and location/URL on the right side of the page
{\href{http://www.getcruise.com}{University of Waterloo}}
{\textsc{}} {
\headedsubsection % Job title entry for the current employer
{Bachelor of Computer Science, Honours, Co-operative Program, AI Specialization}
{Sept '18 -- Apr '23 (expected)}
{\bodytext{
\begin{itemize}
\item 91.26 cumulative GPA, 93.50 major average
\item President's Research Award, President's International Experience Award
\item Six 4-month co-op placements at six distinct companies
\end{itemize}

}}
}
\spacedhrule{0.4em}{-0.4em} % Horizontal rule - the first bracket is whitespace before and the second is after

    \roottitle{\href{https://scholar.google.ca/citations?user=YK20kOEAAAAJ&hl=en&oi=sra}{Publications}} % Top level section
    
    {
{
\headedsubsection % Job title entry for the current employer
{\href{https://arxiv.org/abs/2202.07133}{Sim-to-Real Domain Adaptation for Lane Detection and Classification in Autonomous Driving}}
{June 2022}
{\bodytext{
\begin{itemize}
\item Co-Authored a peer-reviewed \href{https://arxiv.org/abs/2202.07133}{research paper} on sim2real lane detection using the \href{https://carla.org/}{CARLA} simulator, integrating current GAN research with state-of-the-art lane detection models. 
\item Presented the paper at a poster board session during the \href{https://iv2022.com/}{Intelligent Vehicles 2022 symposium}.
% \item Deployed a lane detection model into the production C++ stack, using ROS and Intel OpenVINO, increasing model throughput by 200\%.
% \item Trained and deployed multiple computer vision models for autonomous vehicles, including a a YOLOv3 model for traffic light detection and an EfficientNet-based segmentation model for roadline detection.
\end{itemize}
}
}}

\headedsubsection
{\href{https://www.spiedigitallibrary.org/conference-proceedings-of-spie/11756/117560C/Application-of-machine-learning-for-drone-classification-using-radars/10.1117/12.2588694.short?SSO=1}
{Application of machine learning for drone classification using radars}}
{April 2021}
{\bodytext{
\begin{itemize}
\item Authored a research paper and presented it at the \href{https://www.spiedigitallibrary.org/conference-proceedings-of-spie/11756/117560C/Application-of-machine-learning-for-drone-classification-using-radars/10.1117/12.2588694.short?SSO=1}{SPIE 2021 Conference}, with a follow-up \href{https://www.mdpi.com/2504-446X/5/4/149}{peer-reviewed paper}.
% \item Came 1st place in the \href{https://github.com/SinclairHudson/CANSOFCOM}{Hack The North 2020++ CANSOFCOM Drone ML challenge}.
\item Used a convolutional neural network to classify 5 different commercial drones based on a Fourier transform of their noisy RADAR return signal.
\end{itemize}
}}

\spacedhrule{1.4em}{-0.4em} % Horizontal rule - the first bracket is whitespace before and the second is after

%----------------------------------------------------------------------------------------
%	EXPERIENCE SECTION
%----------------------------------------------------------------------------------------

\roottitle{Experience} % Top level section

\headedsection % Employer name which can include a hyperlink and location/URL on the right side of the page
{\href{http://www.getcruise.com}{Cruise}}
{\textsc{San Francisco, CA}} {
\headedsubsection % Job title entry for the current employer
{Deep Learning Performance Engineer}
{Sept '22 -- Dec '22}
{\bodytext{
\begin{itemize}
\item Developed various Python tools to deploy deep learning models on the vehicle, using \href{https://onnx.ai/}{ONNX}, \href{https://developer.nvidia.com/tensorrt}{TensorRT}, and custom computation graph representations.
\item Built a tool to assess numerical divergence between the original PyTorch models and optimized models.
\item Built a linter to map ONNX nodes to lines in PyTorch source code, saving model deployment engineers hours.
\item Created a tool to manually add outputs to exported ONNX graphs, allowing engineers to inspect intermediate activations while debugging models.
\end{itemize}

}}
}

\headedsection % Employer name which can include a hyperlink and location/URL on the right side of the page
{\href{http://www.nvidia.com}{NVIDIA}}
{\textsc{Santa Clara, CA (remote)}} {
\headedsubsection % Job title entry for the current employer
{Deep Learning Research for Autonomous Vehicles}
{Jan '22 -- Apr '22}
{\bodytext{
\begin{itemize}
\item Designed and iterated on multiple experiments for a LiDAR object detection neural network, improving cyclist and pedestrian F-scores by 43\% and 15\%, respectively. 
\item Implemented sparse tensor object detectors using \href{https://github.com/NVIDIA/MinkowskiEngine}{Minkowski Engine}, outperforming the baseline model while using 70\% less memory.
\item Integrated confidence predictions into a LiDAR object detection auto-labeling pipeline, allowing human annotators to focus efforts on anomalous and challenging data instances.
\end{itemize}

}}
}

\headedsection % Employer name which can include a hyperlink and location/URL on the right side of the page
{\href{https://darwinai.com/}{DarwinAI}}
{\textsc{Waterloo, ON (remote)}} {

\headedsubsection % Job title entry for the current employer
{Machine Learning Developer}
{May '21 -- Aug '21}
{\bodytext{
\begin{itemize}
\item Built and tested defect detection deep learning solutions for clients in the manufacturing industry, focusing on defect detection.
\item Implemented the core functionality of Dataset Distillation using the autograd package, to pursue research in low-data machine learning contexts.
\item Trained XGBoost and SVR systems to model the relationship between environmental conditions and yield for an agriculture client, achieving 11\% median error by weight.
\item Created an anomaly detection research repository in PyTorch, for detecting anomalies in images.
\item Implemented VAE, VQ-VAE, and VQ-VAE-2 from scratch in PyTorch, evaluating each autoencoder as an anomaly detector.
\end{itemize}

}}
}

\headedsection % Employer name which can include a hyperlink and location/URL on the right side of the page
 {\href{https://www.untether.ai/}{Untether AI}}
 {\textsc{Toronto, ON (remote)}} {

\headedsubsection % Job title entry for the current employer
{Software Developer}
{Sept '20 -- Dec '20}
{\bodytext{
\begin{itemize}
\item Built a customer-facing Python API to optimize, format and quantize TensorFlow computation graphs.
\item Designed and implemented Non-Max Suppression for quantized values using only integer operations, allowing Single-Shot Detector pipelines to be run on-chip.
\item Experimented with different quantization schemes to improve the mAP of an SSD-ResNet-34 by 5\%.
 \item Implemented a lookup table class to represent arbitrary non-linear functions in a quantized space.
 \end{itemize}

 }}
 }

\headedsection % Employer name which can include a hyperlink and location/URL on the right side of the page
{\href{http://dev3.noahlab.com.hk/about.html}{Huawei}}
{\textsc{Markham, ON}} {

\headedsubsection % Job title entry for the current employer
{LiDAR Perception Researcher}
{Jan '20 -- Apr '20}
{\bodytext{
\begin{itemize}
    \item Built DBLiDARNet and focal loss from scratch in PyTorch to use in semantic segmentation experiments.
    \item Implemented key modules from 12 different research papers in PyTorch, summarizing state-of-the-art techniques and enabling further research. 
%----------------------------------------------------------------------------------------
    \item Analysed the SemanticKITTI dataset to produce optimal class loss weights, increasing mIoU by 2\%.
    \item Wrote a data loader to spatially align sequential LiDAR scans for temporal pipelines, based on IMU data.
    \end{itemize}
    
    }}
    }
    %------------------------------------------------
    
     \begin{center}
    \textit{Please refer to \href{https://www.linkedin.com/in/sinclairhudson/}{my Linkedin profile} for the complete list of work experiences.}
    \end{center}
    
    %------------------------------------------------
    
    \spacedhrule{0.6em}{-0.4em} % Horizontal rule - the first bracket is whitespace before and the second is after

\roottitle{Skills} % Top level section

\inlineheadsection{Languages: }{ Python, C++, R, C, LaTeX, Java, Javascript}
\inlineheadsection{Frameworks: }{PyTorch, TensorFlow, NumPy, Pandas, OpenCV, ROS}
\inlineheadsection{Tools: }{Git, Docker, Conda, Bazel, CARLA, VIM, GCP, Linux, TensorRT}


\end{document}
